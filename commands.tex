%\usepackage[n, 
%% or lambda advantage , operators , sets,
%adversary , landau , probability , notions ,
%logic ,ff, mm,
%primitives , events , complexity , oracles , asymptotics , keys]{cryptocode}
\usepackage[utf8]{inputenc}
\usepackage{bbm}
\usepackage{xcolor}
\usepackage[utf8]{inputenc} % allow utf-8 input
\usepackage[T1]{fontenc}    % use 8-bit T1 fonts
\usepackage{hyperref}       % hyperlinks
\usepackage{url}            % simple URL typesetting
\usepackage{booktabs}       % professional-quality tables
\usepackage{amsfonts}       % blackboard math symbols
\usepackage{nicefrac}       % compact symbols for 1/2, etc.
\usepackage{microtype}      % microtypography
\usepackage{amsthm}
\usepackage{tikz}
\usepackage{tabularx}
\usepackage[most]{tcolorbox}
\usepackage{cleveref}       % smart cross-referencing
\usepackage{lipsum}         % Can be removed after putting your text content
\usepackage{graphicx}
\usepackage{algorithm}
\usepackage{stmaryrd} % llbracket stuff
%\usepackage{hyperref}
\usepackage[square,sort,comma,numbers]{natbib}
\usepackage{doi}
\graphicspath{ {assets/} }
\usepackage{bbding}
\usepackage[colorinlistoftodos,prependcaption]{todonotes}
\usepackage[n, 
% or lambda advantage , operators , sets,
adversary , landau , probability , notions ,
logic ,ff, mm,
primitives , events , complexity , oracles , asymptotics , keys]{cryptocode}

\usepackage{stmaryrd}

% FONT For The paper
\usepackage{sans}
\usepackage[T1]{fontenc}
\renewcommand\familydefault{\sfdefault}
\usepackage{setspace}
\onehalfspacing


\definecolor{amber}{rgb}{1.0, 0.49, 0.0}
\definecolor{cadmiumgreen}{rgb}{0.0, 0.42, 0.24}
\definecolor{darkcyan}{rgb}{0.0, 0.55, 0.55}
\definecolor{darkcoral}{rgb}{0.8, 0.36, 0.27}
\definecolor{azure}{rgb}{0.0, 0.5, 1.0}
\definecolor{bittersweet}{rgb}{1.0, 0.44, 0.37}
\definecolor{razzmatazz}{rgb}{0.89, 0.15, 0.42}

\hypersetup{
  colorlinks,
  citecolor=cadmiumgreen,
  linkcolor=azure,
  urlcolor=azure}
  



\newtheoremstyle{theorem}%
{3pt}% Space above
{3pt}% Space below 
{}% Body font
{}% Indent amount
{\bfseries\color{razzmatazz}}% Theorem head font
{}% Punctuation after theorem head
{.5em}% Space after theorem head
{}% Theorem head spec (can be left empty, meaning ‘normal’)
\theoremstyle{theorem}
\newtheorem{thm}{Theorem}
\newtheorem{lemma}{Lemma}
\newtheorem{claim}{Claim}
\newtheorem{fact}{Fact}
\newtheorem{corollary}{Corollary}

\newtheoremstyle{definition}%
{3pt}% Space above
{3pt}% Space below 
{}% Body font
{}% Indent amount
{\bfseries\color{darkcyan}}% Theorem head font
{}% Punctuation after theorem head
{.5em}% Space after theorem head
{}% Theorem head spec (can be left empty, meaning ‘normal’)
\theoremstyle{definition}
\newtheorem{definition}{Definition}
\newtheorem{goal}{Goal}


\newtheoremstyle{remark}%
{3pt}% Space above
{3pt}% Space below 
{}% Body font
{}% Indent amount
{\bfseries\color{razzmatazz}}% Theorem head font
{}% Punctuation after theorem head
{.5em}% Space after theorem head
{}% Theorem head spec (can be left empty, meaning ‘normal’)
\theoremstyle{remark}
\newtheorem{remark}{Remark}


%\newcommand{\stateclaimsolid}[2]{
%  \par\noindent\tikzstyle{mybox} = [draw=black,fill=yellow!20,
%   thick,rectangle,rounded corners, inner sep=6pt,path picture={\fill ([xshift=-6.15cm]path picture bounding box.north) rectangle (path picture bounding box.south west);}]
%  \begin{tikzpicture}
%   \node [mybox] (box){%
%    \begin{minipage}{#1}{#2}\end{minipage}
%   };
%  \end{tikzpicture}
%}

\newcommand{\stateboxsolid}[1]{
  \par\noindent\tikzstyle{mybox} = [draw=black,fill=white!20,
   thick,rectangle,rounded corners,inner sep=6pt]
  \begin{tikzpicture}
    \node [mybox] (box){%
    \begin{minipage}{0.99\textwidth}{#1}\end{minipage}
   };
  \end{tikzpicture}
}
\newcommand{\statetheoremsolid}[2]{
  \par\noindent\tikzstyle{mybox} = [draw=black,fill=yellow!20,
   thick,rectangle,rounded corners,inner sep=6pt]
  \begin{tikzpicture}
    \node [mybox] (box){%
    \begin{minipage}{#1}{#2}\end{minipage}
   };
  \end{tikzpicture}
}
% TIKZ stuff above

% Typical re-transformations
%\renewcommand{\epsilon}{\varepsilon}
\newcommand{\bit}{\{ 0, 1\}}
\renewcommand{\poly}{\mathsf{poly}}
%\newcommand{\pp}{\mathtt{pp}}
\newcommand{\samples}{\xleftarrow{\$}}
\newcommand{\inputs}{\xleftarrow{\mathsf{input}}}
\newcommand{\outputs}{\xrightarrow{\mathsf{output}}}
\newcommand{\samplesIdd}{\xleftarrow{\mathsf{i.i.d}}}

\newcommand{\AdvA}{\ensuremath{\mathcal{A}}}
\renewcommand{\oracle}{\ensuremath{\mathcal{O}}}

\newcommand{\PermDist}{\mathcal{D}}
\newcommand{\FarPermDist}{\hat{\mathcal{D}}}


\newcounter{protocol}
\newenvironment{protocol}[1]
  {\par\addvspace{\topsep}
   \noindent
   \tabularx{\linewidth}{@{} X @{}}
    \hline
    \refstepcounter{protocol}\textbf{Protocol \theprotocol} #1 \\
    \hline}
  { \\
    \hline
   \endtabularx
   \par\addvspace{\topsep}}

\newcommand{\xmark}{\text{}}


\newcommand{\etal}{\textit{et al.}{ }}
\newcommand{\bin}{\mathtt{Bin}}

\newcommand{\Z}{\mathbb{Z}}
\newcommand{\F}{\mathbb{F}}
\newcommand{\R}{\mathbb{R}}
\newcommand{\G}{\mathbb{G}}
\renewcommand{\pp}{\texttt{pp}}
\newcommand{\noisen}{\eta}
\newcommand{\Mechanism}{\mathtt{M}}

\renewcommand{\OR}{\mathsf{OR}}
\newcommand{\OrProtocol}{\Pi_\OR}
\newcommand{\BitLang}{\mathsf{L}_{\mathsf{Bit}}}
\newcommand{\Setup}{\mathtt{Setup}}
\newcommand{\Morra}{\mathtt{morra}}
\newcommand{\Output}{\mathtt{out}}
\newcommand{\share}[1]{\llbracket #1 \rrbracket}
\renewcommand{\gets}{\xleftarrow{\texttt{Input}}}
%\newcommand{\SecretShare}[2]{\Big( [#1]_{1}, \dots, [#1]_{#2}\Big)}

% Commitment stuff
\newcommand{\RandomnessSpace}{\mathsf{Rand}}
\newcommand{\MessageSpace}{\mathsf{Msg}}


\DeclareMathOperator{\Com}{Com}
\renewcommand{\bin}{\mathsf{Bin}}
%text symbols
\newcommand{\SecurityParam}{\kappa}
\newcommand{\Naturals}{\mathbb{N}}
\newcommand{\Normal}{\mathcal{N}}
\newcommand{\Integers}{\mathbb{Z}}
\newcommand{\Reals}{\mathbb{R}}
\newcommand{\Identity}{\mathit{I}}
\newcommand{\Exp}{\mathsf{exp}}
\renewcommand{\exp}{\Exp}
\newcommand{\Eps}{\varepsilon}
\newcommand{\EpsDelta}{(\Eps, \delta)}

%functions
\newcommand{\FuncDomain}{\mathcal{X}}
\newcommand{\FuncRange}{\mathcal{Y}}
\newcommand{\FuncFamily}{\mathcal{Q}}
\DeclareMathOperator*{\argmax}{arg\,max}
\DeclareMathOperator*{\argmin}{arg\,min}

%graphs
\newcommand{\Vertices}{\mathcal{V}}
\newcommand{\Edges}{\mathcal{E}}
\newcommand{\Adjacency}{\mathtt{Adj}}
\newcommand{\Incidence}{\mathtt{Inc}}
\newcommand{\hamming}[1]{d_\mathtt{Ham} #1}

%distributions
\newcommand{\Dist}{\mathcal{D}}
\newcommand{\Support}{\mathsf{Support}}

\newcommand{\SuppSize}{\mathsf{n}}
\newcommand{\DomainDist}{\Omega_{\SuppSize}}
\newcommand{\DomainDistJoint}{\Omega_{\mathsf{n_1} \times \mathsf{n_2}}}
\newcommand{\DomainDistJointSame}{\Omega_{\mathsf{n} \times \mathsf{n}}}
\newcommand{\DistSet}{\Delta(\DomainDist)}
\newcommand{\DistSetJoint}{\Delta(\DomainDistJoint)}
\newcommand{\DistSetJointSame}{\Delta(\DomainDistJointSame)}
\newcommand\TotalVar[2]{d_{\textsf{TV}}(#1,#2)}
\newcommand{\Negl}{\mu}
\newcommand{\AvgMinEntropy}{\tilde{H}_\infty}


%distribution testing
\newcommand{\DIP}{\mathsf{DIP}}
\newcommand{\EZKDIP}{\mathsf{EZKDIP}}
\newcommand{\ZKDIP}{\mathsf{ZKDIP}}
\newcommand{\SZKDIP}{\mathsf{SZKDIP}}
\newcommand{\HVCZK}{\mathsf{HVCZK}}
\newcommand{\HVCZKD}{\mathsf{HVCZK-D}}
\newcommand{\out}{\mathsf{out}}
\newcommand{\Tester}{\mathsf{T}}
\newcommand{\Verifier}{\mathsf{V}}
\newcommand{\Prover}{\mathsf{P}}
\newcommand{\Sender}{\mathsf{S}}
\newcommand{\Receiver}{\mathsf{R}}
\newcommand{\ChSender}{\mathsf{\hat{S}}}
\newcommand{\ChReceiver}{\mathsf{\hat{R}}}
\newcommand{\ChVerifier}{\mathsf{\hat{V}}}
\newcommand{\ChProver}{\mathsf{\widetilde{P}}}
\newcommand{\OptProver}{\mathsf{P}^*}
\newcommand{\Simulator}{\mathsf{Sim}}
\newcommand{\AuxInput}{z}
\newcommand{\View}{\mathrm{view}}
\newcommand{\Prop}{\Pi}
\newcommand{\PropJoint}{\Pi_{\mathsf{n_1} \times \mathsf{n_2}}}
\newcommand{\PropJointSame}{\Pi_{\mathsf{n} \times \mathsf{n}}}
\newcommand{\Proximity}{\delta}
\newcommand{\Accept}{\textsf{accept}}
\newcommand{\Reject}{\textsf{reject}}
\newcommand{\interacton}[2]{ \left(#1, #2 \right)}
\newcommand{\VerifierLocalRand}{\vec{\rho}}
\newcommand{\TV}[1]{d_\mathtt{TV} #1}
\newcommand{\EM}[1]{d_\mathtt{EM} #1}
\newcommand{\statInd}{\stackrel{s}{\equiv}}
\newcommand{\compInd}{\stackrel{c}{\equiv}}


%permuatations
\newcommand{\PermProp}{\Prop^\mathsf{BIJECT}_n}
\newcommand{\PermSize}{n}
\newcommand{\Perm}{f}
\newcommand{\Bad}{\mathsf{Bad}}
\newcommand{\Good}{\mathsf{Good}}


%permutation protocol
\newcommand{\NumSamples}{k}
\newcommand{\NumRounds}{T}
\newcommand{\RandRound}{t}
\newcommand{\RoundInd}{\mathsf{round}}
\newcommand{\PermSamplePoint}{x}
\newcommand{\PermChallengePoint}{y}
\newcommand{\Challenge}{\mathsf{CHALLENGE}}
\newcommand{\NotPerm}{\hat{f}}
\newcommand{\PermProver}{\mathsf{P}_{\text{perm}}}
\newcommand{\PermVerifier}{\mathsf{V}_{\text{perm}}}

\newcommand{\Cheat}{\mathsf{Cheat}}
\newcommand{\True}{\mathsf{TRUE}}
\newcommand{\No}{\mathsf{No}}
\newcommand{\Yes}{\mathsf{Yes}}
\newcommand{\False}{\mathsf{FALSE}}
\newcommand{\calR}{\mathcal{R}}

% computational ZK
\newcommand{\com}{\mathtt{com}}

%mathsf
\newcommand{\sfA}{\mathsf{A}}


%complexity classes
\newcommand{\eIPP}{\ensuremath{\Proximity\mathsf{-IPP}}}

\newcommand{\sbline}{\\[.5\normalbaselineskip]}% small blank line
\newcommand\inner[2]{\langle #1, #2 \rangle}
\newcommand\Biginner[2]{\Big\langle #1, #2 \Big\rangle}
\newcommand{\boocube}{\{0, 1\}^n}
\newcommand{\boodist}{\Delta(\boocube,\boocube)}

%operators
\newcommand{\pST}{\; \middle| \;}

%formatting
\newcommand{\DoQuote}[1]{``#1''}
\newcommand{\defemph}[1]{\textbf{\emph{#1}}}


%notes
\newcommand{\ariInline}[1]{\todo[backgroundcolor=azure!15, bordercolor=azure, textcolor=black, inline]{#1 - Ari}}
\newcommand{\ari}[1]{\todo[backgroundcolor=azure!15, bordercolor=azure, textcolor=black]{#1 - Ari}}

\newcommand{\NR}[1]{\todo[backgroundcolor=bittersweet!15, bordercolor=bittersweet, textcolor=black, inline]{#1 - Ninad}}

\newcommand{\highlight}[1]{\color{razzmatazz}\textit{#1 }\color{black}}
%\newtheorem{definition}{Definition}[section]
%\newtheorem{remark}[definition]{Remark}
%\newtheorem{theorem}[definition]{Theorem}
%\newtheorem{lemma}[definition]{Lemma}
%\newtheorem{corollary}[definition]{Corollary}
%\newtheorem{prop}[definition]{Proposition}
